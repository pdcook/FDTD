\documentclass[12pt,twocolumn]{article}
\usepackage{amsmath,amsfonts,amssymb}
\usepackage{empheq}
\usepackage{setspace}
\usepackage{float}
\usepackage{tocloft}
\usepackage{cuted}
\usepackage{flushend}
\usepackage[scaled]{beramono}
\usepackage[T1]{fontenc}
\usepackage{cite}
\usepackage{amsthm}
\usepackage{authblk}
\usepackage[toc,page]{appendix} % for appendices
\usepackage{gensymb}            % for degree symbol
\usepackage[pdftex]{graphicx}   % for figures
\usepackage{siunitx}            % for si units
\usepackage[caption=false]{subfig}  % for subfigures
\usepackage{algorithm}
\usepackage{algpseudocode}
\usepackage{titling}
\usepackage{hyperref}
\usepackage{epstopdf}
\usepackage{esint}
\usepackage{physics}
\hypersetup{
    colorlinks=true,
    urlcolor=blue,
    linkcolor=blue,
    citecolor=red}
\providecommand{\keywords}[1]{\textbf{\textbf{Keywords:~}} #1}
\captionsetup[subfigure]{labelformat=brace} % setup subfigure captions
\graphicspath{{./Figures/}}     % figures in the Figures directory
\DeclareGraphicsExtensions{.eps,.ps,.pdf}    % extensions for figures
\newtheorem{theorem}{Theorem}
\newtheorem{lemma}{Lemma}
\newcommand{\comment}[1]{}      % command for multiline comments
\setlength\stripsep{3pt plus 1pt minus 1pt}
\renewcommand{\cftdotsep}{\cftnodots}
\cftpagenumbersoff{figure}
\cftpagenumbersoff{table}
\begin{document}
\title{A Finite Difference Time Domain Method for Two Dimensional Electromagnetic Wave Propagation In Heterogeneous Lossy Dielectric Media With Cyclic and Perfectly Matched Layer Boundary Conditions}
\author{Patrick D. Cook}
\affil{Department~of~Physics,~Fort~Hays~State~University,\\Hays,~Kansas~67601,~USA}
\date{Fall 2019}
\begin{titlingpage}
\maketitle
\begin{abstract}
In this work, the foundations behind the one- and two-dimensional finite-difference time-domain (FDTD) method in the context of electromagnetic wave propagation are presented. Heterogeneous propagation media with arbitrary dielectric constants and arbitrary electrical conductivities are considered. Two different boundary conditions, cyclic and perfectly matched layer (PML), are presented and discussed. This work includes implementations of the method using both boundary conditions, written in standard Python 3. These implementations are used to explore a variety of simple and complex geometries.
\end{abstract}

\keywords{Electromagnetic wave simulation, finite-difference time-domain, perfectly matched layer}
\end{titlingpage}
\newpage
\section{Introduction}
The goal of the finite-difference time-domain (FDTD) method is to simulate the propagation of electromagnetic waves in complex geometry. For instance, propagation through several different materials with different dielectric constants, or electrical conductivities. Problems such as these are often impossible to solve analytically, and so numerical approaches, such as this one, are required.

This work is an overview of the derivation of the necessary expressions used in the FDTD method as well as two types of boundary conditions. Implementations of the method for both boundary conditions with results are presented and discussed.
\section{Governing Equations for Electromagnetic Waves}
The differiential form of Maxwell's equations for the electric field, $\mathbf{E}$, and magnetic field, $\mathbf{B}$, in vacuum, are~\cite{Jackson99}
\begin{subequations}
\begin{align}
\div{\mathbf{E}} &= \frac{\rho}{\epsilon_0}, \\
\div{\mathbf{B}} &= 0, \\
\label{eq:curl_E}\curl{\mathbf{E}}&= -\frac{\partial\mathbf{B}}{\partial t}, \\
\label{eq:curl_B}\curl{\mathbf{B}}&= \mu_0\mathbf{J}+\frac{1}{c^2}\frac{\partial\mathbf{E}}{\partial t}.
\end{align}
\end{subequations}
Where
\begin{itemize}
\item $\rho$ is the free electric charge density
\item $\mathbf{J}$ is the free current density
\item $\epsilon_0$ is the permittivity of free space
\item $\mu_0$ is the permeability of free space
\item $c$ is the speed of light, $c\equiv\frac{1}{\sqrt{\epsilon_0\mu_0}}$
\end{itemize}
In the context of electromagnetic wave propagation, only Eqs.~\ref{eq:curl_E}~and~\ref{eq:curl_B} are of interest. Assuming there is no free current, rewriting these two equations in terms of the magnetic field strength, $\mathbf{H}=\mathbf{B}/\mu_0$, and subsituting $c\equiv\frac{1}{\sqrt{\epsilon_0\mu_0}}$ yields
\begin{subequations}
\begin{align}
\label{eq:dH_dt_vacuum}\frac{\partial\mathbf{H}}{\partial t} &= -\frac{1}{\mu_0}\curl{\mathbf{E}}, \\
\label{eq:dE_dt_vacuum}\epsilon_0\frac{\partial\mathbf{E}}{\partial t} &= \curl{\mathbf{H}}.
\end{align}
\end{subequations}
In a dielectric medium, we introduce the relative permittivity, $\epsilon_r$, which is material-specific. This allows us to write the permittivity as $\epsilon=\epsilon_r\epsilon_0$. We could also introduce the relative permeability, but in this work we will assume all materials are nonmagnetic, meaning $\mu_r=1$ and so $\mu=\mu_0$. Eqs.~\ref{eq:dH_dt_vacuum}~and~\ref{eq:dE_dt_vacuum} become
\begin{subequations}
\begin{align}
\frac{\partial\mathbf{H}}{\partial t} &= -\frac{1}{\mu_0}\curl{\mathbf{E}}, \\
\frac{\partial\mathbf{E}}{\partial t} &= \frac{1}{\epsilon}\curl{\mathbf{H}}.
\end{align}
\end{subequations}

A more general form of these equations uses the electric displacement, $\mathbf{D}$, which can be written as $\mathbf{D}(\omega) = \epsilon_0\epsilon_r^*(\omega)\mathbf{E}(\omega)$, where $\omega$ is the frequency of the wave and $\epsilon^*_r$ is the frequency dependent dielectric constant of the medium~\cite{Sullivan00}. If we assume, for a lossy dielectric medium with relative permittivity $\epsilon_r$ and conductivity $\sigma$, that $\epsilon^*_r$ is of the form
\begin{equation}
\epsilon^*_r(\omega) = \epsilon_r + \frac{\sigma}{i\omega\epsilon_0},
\end{equation}
then we see that the electric displacement is
\begin{equation}
\mathbf{D}(\omega) = \epsilon_0\epsilon_r \mathbf{E}(\omega) + \frac{\sigma}{i\omega}\mathbf{E}(\omega).
\end{equation}
Transforming this from the frequency domain to the time domain requires an identity which is derived in Appendix~\ref{ap:fourier}. In the time domain, the above equation is
\begin{equation}
\mathbf{D}(t) = \epsilon_0\epsilon_r \mathbf{E}(t)+\sigma\int_0^t\mathbf{E}(\tau)d\tau.
\end{equation}
So that the relevant equations for electromagnetic waves are then
\begin{subequations}
\begin{align}
\label{eq:dH_dt_SI}\frac{\partial\mathbf{H}}{\partial t} &= -\frac{1}{\mu_0}\curl{\mathbf{E}}, \\
\label{eq:D_time_SI}\mathbf{D}(t) &= \epsilon_0\epsilon_r \mathbf{E}(t)+\sigma\int_0^t\mathbf{E}(\tau)d\tau, \\
\label{eq:dD_dt_SI}\frac{\partial\mathbf{D}}{\partial t} &= \curl{\mathbf{H}}.
\end{align}
\end{subequations}

Finally, we express everything in Gaussian units instead of SI\@. The purpose for this is so that $E$, $D$, and $H$ are around the same order of magnitude~\cite{Sullivan00}. In Gaussian units,
\begin{subequations}
\begin{align}
\mathbf{\tilde{E}}=\sqrt{\frac{\epsilon_0}{\mu_0}}\mathbf{E},\\
\mathbf{\tilde{D}}=\sqrt{\frac{1}{\epsilon_0\mu_0}}\mathbf{D}.
\end{align}
\end{subequations}
This normalization yields the final set of equations that will be used to govern the propagation of electromagnetic waves in a lossy dielectric medium,
\begin{subequations}
\begin{empheq}[box=\fbox]{align}
\label{eq:dH_dt}\frac{\partial\mathbf{H}}{\partial t} &= -\frac{1}{\sqrt{\epsilon_0\mu_0}}\curl{\mathbf{\tilde{E}}}, \\
\label{eq:D_time}\mathbf{\tilde{D}}(t) &= \epsilon_r \mathbf{\tilde{E}}(t)+\frac{\sigma}{\epsilon_0}\int_0^t\mathbf{\tilde{E}}(\tau)d\tau, \\
\label{eq:dD_dt}\frac{\partial\mathbf{\tilde{D}}}{\partial t} &= \frac{1}{\sqrt{\epsilon_0\mu_0}}\curl{\mathbf{H}}.
\end{empheq}
\end{subequations}
In these equations, the two parameters that are material-specific appear only in Eq.~\ref{eq:D_time}. Larger relative permittivities, $\epsilon_r$, decrease the strength of the electric field per unit charge, which has the effect of slowing electromagnetic waves. The dielectric conductivity, $\sigma$, represents the total dissipative effect of the material, accounting for any energy loss of the waves in the material.
\section{Finite Difference Approximation of the Derivative}
Let $f(x)$ be any continuous and thrice differientiable function with bounded derivatives. If we want to obtain an approximation of $f'(x)$, we start by considering $f(x+h)$ and $f(x-h)$, where $h$ is a small displacement. Theorem~\ref{thm:lagrange} in Appendix~\ref{ap:lagrange} allows us to write $f(x+h)$ and $f(x-h)$ as shown in Eq.~\ref{eq:expansions}.
\vspace{-20pt}
\begin{strip}
\begin{equation}
\label{eq:expansions}
\begin{aligned}
f(x+h) &= f(x) + hf'(x) + \frac{h^2}{2}f''(x) + \frac{h^3}{3!}f'''(\eta_1) \\
f(x-h) &= f(x) -hf'(x) + \frac{h^2}{2}f''(x) -\frac{h^3}{3!}f'''(\eta_2)
\end{aligned}
\end{equation}
\end{strip}
\begin{subequations}
\begin{strip}
\begin{equation}
\label{eq:first_finite}
\frac{f(x+h)-f(x-h)}{2h} = f'(x) + \frac{h^2}{6}\frac{f'''(\eta_1)+f'''(\eta_2)}{2}
\end{equation}
\begin{equation}
\label{eq:second_finite}
\frac{f(x+h)-f(x-h)}{2h} = f'(x) + \frac{h^2}{6}f'''(\eta)
\end{equation}
\end{strip}
\end{subequations}

\noindent Subtracting both expressions in Eq.~\ref{eq:expansions} and rearranging yields Eq.~\ref{eq:first_finite}. Here, $\eta_1$ and $\eta_2$ are two unknown points in the domain of $f$. By applying the intermediate value theorem to the last term in Eq.~\ref{eq:first_finite}, the two unknown points can be reduced to a single unknown point, $\eta$. This leads to Eq.~\ref{eq:second_finite}. This equation suggests an approximation for the first derivative,
\begin{equation}
\label{eq:approx_central_difference}
f'(x) \approx \frac{f(x+h)-f(x-h)}{2h}.
\end{equation}
With an error of
\begin{equation}
|E| = \left|\frac{h^2}{6}f'''(\eta)\right|.
\end{equation}

Since $f$ and all derivatives of $f$ are bounded, we may introduce $M$, the absolute maxmimum value of $f'''(x)$, meaning $|f'''(x)|\leq M~\forall~x$. Thus, the error term is a constant multiplied by $h^2$. We say then that this error is an order of $h^2$, or
\begin{equation}
\label{eq:central_difference}
f'(x) = \frac{f(x+h)-f(x-h)}{2h} + O(h^2).
\end{equation}
This is known as the central difference approximation of the derivative. Due to the error being in $h^2$, this is a second order approximation. There are many other approximations of the first derivative as well as higher order derivatives. However, this is the only derivative approximation necessary in FDTD.
\section{Formulation of FDTD in One Dimension}
For example purposes, a formulation of the FDTD method in one dimension is presented here. First, we must choose an axis of propagation. Arbitrarily, suppose the wave travels only in the $\hat{z}$ direction. Eqs.~\ref{eq:dH_dt}~and~\ref{eq:dD_dt} become
\begin{subequations}
\begin{align}
\frac{\partial H_y}{\partial t} &= -\frac{1}{\sqrt{\epsilon_0\mu_0}}\frac{\partial \tilde{E_x}}{\partial z}, \\
\frac{\partial \tilde{D_x}}{\partial t} &= -\frac{1}{\sqrt{\epsilon_0\mu_0}}\frac{\partial H_y}{\partial z}.
\end{align}
\end{subequations}
We will interleave $H_y$ and $\tilde{D_x}$ both spatially and temporally. This is one of the fundamental features of the FDTD method~\cite{Sullivan00}. To interleave the two fields, we evaluate them---as well as their derivatives---offset from one another by $\Delta z/2$ in space and $\Delta t/2$ in time,
\begin{subequations}
\begin{equation}
\label{eq:offset_dD_dt}\frac{\partial \tilde{D_x}}{\partial t}(z, t) = -\frac{1}{\sqrt{\epsilon_0\mu_0}}\frac{\partial H_y}{\partial z}(z, t)
\end{equation}
\vspace{-65pt}
\begin{strip}
\begin{equation}
\label{eq:offset_dH_dt}\frac{\partial H_y}{\partial t}\left(z+\frac{\Delta z}{2},t+\frac{\Delta t}{2}\right) = -\frac{1}{\sqrt{\epsilon_0\mu_0}}\frac{\partial \tilde{E_x}}{\partial z}\left(z+\frac{\Delta z}{2},t+\frac{\Delta t}{2}\right)
\end{equation}
\end{strip}
\end{subequations}
\begin{subequations}
\begin{strip}
\begin{align}
\label{eq:first_discrete_dD_dt}\frac{\tilde{D_x}\left(z,t+\frac{\Delta t}{2}\right)-\tilde{D_x}\left(z,t-\frac{\Delta t}{2}\right)}{\Delta t} &= -\frac{1}{\sqrt{\epsilon_0\mu_0}}\frac{H_y\left(z+\frac{\Delta z}{2},t\right)-H_y\left(z-\frac{\Delta z}{2},t\right)}{\Delta z} \\
\label{eq:first_discrete_dH_dt}\frac{H_y\left(z+\frac{\Delta z}{2},t+\Delta t\right)-H_y\left(z+\frac{\Delta z}{2},t\right)}{\Delta t} &= -\frac{1}{\sqrt{\epsilon_0\mu_0}}\frac{\tilde{E_x}\left(z+\Delta z,t+\frac{\Delta t}{2}\right)-\tilde{E_x}\left(z,t+\frac{\Delta t}{2}\right)}{\Delta z}
\end{align}
\end{strip}
\end{subequations}
Using the central difference approximation of the derivative, Eq.~\ref{eq:approx_central_difference}, on each of the derivatives---using the same spacing as the offsets from the interleaving process, $\Delta z/2$ for the spatial stepsize and $\Delta t/2$ for the temporal stepsize---leads us to Eqs.~\ref{eq:first_discrete_dD_dt}~and~\ref{eq:first_discrete_dH_dt}.

Now, we impose restrictions on $\Delta t$ and $\Delta z$ to ensure stability. A discussion including the reasoning for this is presented in the next section. By forcing
\begin{equation}
\Delta t=\frac{\Delta z}{2c_0},
\end{equation}
 where $c_0=1/\sqrt{\epsilon_0\mu_0}$ just as before, we guarantee the stability of the numerical method. Using this condition and rearranging Eqs.~\ref{eq:first_discrete_dD_dt}~and~\ref{eq:first_discrete_dH_dt} into an iterative form leads to
\begin{subequations}
\begin{equation}
\label{eq:second_discrete_dD_dt}\begin{split}
\tilde{D_x}\left(z,t+\frac{\Delta t}{2}\right)&=\tilde{D_x}\left(z,t-\frac{\Delta t}{2}\right) \\
&   -\frac{1}{2}\left[H_y\left(z+\frac{\Delta z}{2},t\right)\right. \\
&- \left. H_y\left(z-\frac{\Delta z}{2},t\right)\right],
\end{split}
\end{equation}
\begin{equation}
\label{eq:second_discrete_dH_dt}\begin{split}
H_y\left(z+\frac{\Delta z}{2},t+\Delta t\right)=H_y\left(z+\frac{\Delta z}{2},t\right) \\
     -\frac{1}{2}\left[\tilde{E_x}\left(z+\Delta z,t+\frac{\Delta t}{2}\right)\right. \\
\left. -\tilde{E_x}\left(z,t+\frac{\Delta t}{2}\right)\right].
\end{split}
\end{equation}
\end{subequations}
Equation~\ref{eq:second_discrete_dD_dt} tells us the entire $\tilde{D_x}$ field at the next time step if both $\tilde{D_x}$ and $H_y$ are known at the previous time steps. Likewise, Eq.~\ref{eq:second_discrete_dH_dt} tells us the entire $H_y$ field at the next time step if $\tilde{E_x}$ and $H_y$ are known at the previous time steps. We can discretize these expressions by defining
\begin{equation}
\begin{aligned}
\left\{
\begin{split}
z_i&=i\Delta z \\
t_j&=\left(j+\frac{1}{2}\right)\Delta t
\end{split}\right.~~&~\mathrm{for}~\tilde{E_x}~\mathrm{and}~\tilde{D_x}, \\
\left\{
\begin{split}
z_i&=\left(i+\frac{1}{2}\right)\Delta z \\
t_j&=j\Delta t
\end{split}\right.~~&~\mathrm{for}~H_y.
\end{aligned}
\end{equation}
Which results in much more succinct expressions,
\begin{subequations}
\begin{align}
\label{eq:third_discrete_dD_dt}\begin{split}
&\tilde{D_x}(z_i,t_{j})=\tilde{D_x}(z_i,t_{j-1}) \\
&   -\frac{1}{2}\left[H_y(z_{i},t_j)-H_y(z_{i-1},t_j)\right],
\end{split} \\
\label{eq:third_discrete_dH_dt}\begin{split}
&H_y(z_i,t_{j+1}) =H_y(z_i,t_{j}) \\
    & -\frac{1}{2}\left[\tilde{E_x}(z_{i+1},t_j)-\tilde{E_x}(z_{i},t_j)\right].
\end{split}
\end{align}
\end{subequations}

 However, we still need a way to find $\tilde{E_x}$ at the next timestep. For this, we return to Eq.~\ref{eq:D_time}. The discrete form of the integral is just a summation, so the discrete form of Eq.~\ref{eq:D_time} is
\begin{equation}
\label{eq:first_discrete_D_time}\tilde{D_x}(z_i,t_j)=\epsilon_r\tilde{E_x}(z_i,t_j)+\frac{\sigma}{\epsilon_0}\sum_{n=0}^{j}\tilde{E_x}(z_i,t_n)\Delta t.
\end{equation}
By separating the $j^\mathrm{th}$ term from the rest of the summation, we see that
\begin{equation}
\begin{split}
\label{eq:second_discrete_D_time}\tilde{D_x}(z_i,t_j)&=\epsilon_r\tilde{E_x}(z_i,t_j)+ \frac{\sigma\Delta t}{\epsilon_0}\tilde{E_x}(z_i,t_j) \\
&+\frac{\sigma\Delta t}{\epsilon_0}\sum_{n=0}^{j-1}\tilde{E_x}(z_i,t_n).
\end{split}
\end{equation}
This allows us to find $\tilde{E_x}(z_i,t_j)$, which is $\tilde{E}$ at the current time, from $\tilde{D_x}$ at the current time as well as $\tilde{E_x}$ at all past times,
\begin{equation}
\label{eq:first_discrete_E}
\tilde{E_x}(z_i,t_j) = \frac{\tilde{D_x}(z_i,t_j)-\frac{\sigma\Delta t}{\epsilon_0}\sum_{n=0}^{j-1}\tilde{E_x}(z_i,t_j)}{\epsilon_r+\frac{\sigma\Delta t}{\epsilon_0}}.
\end{equation}
For conciseness as well as efficiency in implementation, we will introduce
\begin{equation}
\label{eq:first_I}
I(z_i,t_j) = \frac{\sigma\Delta t}{\epsilon_0}\sum_{n=0}^{j}\tilde{E_x}(z_i,t_n).
\end{equation}
This allows us to forgo the need to perform the entire summation each time, instead simply adding the newest value of $\tilde{E_x}$ each time instead.

With the substitution of Eq.~\ref{eq:first_I} in Eq.~\ref{eq:first_discrete_E} alongside Eqs.~\ref{eq:third_discrete_dD_dt}~and~\ref{eq:third_discrete_dH_dt}, we arrive at the formulation of FDTD in one dimension, arranged in iterative form,
\begin{subequations}
\begin{empheq}[box=\fbox]{align}
\label{eq:1D_D}\begin{split}
&\tilde{D_x}(z_i,t_{j})=\tilde{D_x}(z_i,t_{j-1}) \\
&   -\frac{1}{2}\left[H_y(z_{i},t_j)-H_y(z_{i-1},t_j)\right],
\end{split} \\
\label{eq:1D_E}
&\tilde{E_x}(z_i,t_j) = \frac{\tilde{D_x}(z_i,t_j)-I(z_i,t_{j-1})}{\epsilon_r+\frac{\sigma\Delta t}{\epsilon_0}},\\
\label{eq:1D_I}
&I(z_i,t_j) = \frac{\sigma\Delta t}{\epsilon_0}\sum_{n=0}^{j}\tilde{E_x}(z_i,t_n), \\
\label{eq:1D_H}\begin{split}
&H_y(z_i,t_{j+1}) =H_y(z_i,t_{j}) \\
    & -\frac{1}{2}\left[\tilde{E_x}(z_{i+1},t_j)-\tilde{E_x}(z_{i},t_j)\right].
\end{split}
\end{empheq}
\end{subequations}
Again, an important observation is that only Eqs.~\ref{eq:1D_E}~and~\ref{eq:1D_I} contain information about the propagation medium. Thus, Eqs.~\ref{eq:1D_D}~and~\ref{eq:1D_H} do not change in different media.
\section{Stability}
The stability conditions of FDTD, unlike many numerical methods, comes from physical reasoning rather than mathematical reasoning. FDTD propagates electromagnetic waves, which clearly cannot move faster than $c_0$, the speed of light in vacuum. As such, imagine an electromagnetic wave propagating in one dimension. If the distance it travels is $\Delta x$, then the minimum time required is $\Delta t_{\textit{min}} = \Delta x/c_0$. Suppose this distance, $\Delta x$, is the length of one cell in the one-dimensional FDTD, then the timestep, $\Delta t$, must be no greater than $\Delta t_\textit{min}$. So in one dimension, $\Delta t \leq \Delta x/c_0$. Now, consider a two-dimensional FDTD, now the maximum distance the wave could travel in one cell is the diagonal of each square cell. This leads us to a requirement of $\Delta t \leq \frac{\Delta x}{c_0\sqrt{2}}$ where $\Delta x$ is the size of the cell in both spatial dimensions. This trend continues, leading us to the Courant-Friedrichs-Levy (CFL) condition for an explicit method~\cite{Courant28},
\begin{equation}
\Delta t \leq \frac{1}{c_0\sqrt{\sum_i\frac{1}{(\Delta x_i)^2}}},
\end{equation}
where the sum in the denominator is over all spatial dimensions. For example, the stability condition for a three-dimensional FDTD would be $\Delta t \leq \frac{\Delta x}{c_0\sqrt{3}}$ if the size of the cells were $\Delta x \times \Delta x \times \Delta x$. For $n$ dimensions with equal spacing,
\begin{equation}
\Delta t \leq \frac{\Delta x}{c_0\sqrt{n}}.
\end{equation}

For simplicity in this paper, we choose
\begin{equation}
\Delta t = \frac{\Delta x}{2 c_0},
\end{equation}
which guarantees stability in one-, two-, three-, and even four-dimensional FDTD simulations. However, this is not the fastest or most efficient choice of $\Delta t$.

\section{Boundary Conditions}
In code, we cannot simulate infinite fields. This applies not only to the one-dimensional method, but all higher dimensional methods as well. As such, we will talk about $x$ as a general spatial coordinate here, instead of $z$. Since we cannot store fields with infinite size in any computer code, there must be some finite number of $x$ values, say $N$. Then there must be some initial and final $x_i$, say $x_0$ and $x_{N-1}$ respectively. This section discusses three techniques to handle the non-existent values of $x_{-1}$ and $x_{N}$.
\subsection{Hard Boundaries}
One of the simplest ways to handle the boundaries is to force ${x_i=0~\mathrm{for}~i<0~\mathrm{or}~i>N-1}$. However, this results in erroneous reflections, which are not ideal~\cite{Sullivan00}.
\subsection{Cyclic Boundaries}
A similar, but slightly more efficient boundary condition is the cyclic boundary condition in which
\begin{equation}
\label{eq:cyclic}
i=\begin{cases}
i+N\left\lceil\left|{\frac{i}{N}}\right|\right\rceil~&\mathrm{if}~i<0,\\
i~&\mathrm{if}~0\leq i\leq N-1,\\
i-N\left\lfloor\left|{\frac{i}{N}}\right|\right\rfloor~&\mathrm{if}~i>N-1.
\end{cases}
\end{equation}
Where $\lceil~\rceil$ and $\lfloor~\rfloor$ are the ceiling and floor functions respectively. This condition makes it so that leaving one side of the computational domain results in appearing on the other.

As an example, let $N=100$ and suppose the values of $x_{128}$ and $x_{-128}$ are required. In the first case, we are in the bottom case of Eq.~\ref{eq:cyclic}, which evaluates to $i=128-100\lfloor 1.28 \rfloor=28$. Likewise, for $x_{-128}$, $i=-128+100\lceil1.28\rceil=72$. And so $x_{128}\rightarrow x_{28}$ and $x_{-128}\rightarrow x_{72}$---values that exist within the domain.

In terms of erroneous propagation, this method is not necessarily any better than the previous one. Instead of erroneous reflections, now certain parts of the boundary act as sources, assuming a wave propagated through the opposing side.
\subsection{Perfectly Matched Layer}

\section{Heterogeneous Media}

\section{Formulation of FDTD in Two Dimensions}



\bibliography{References}
\bibliographystyle{ieeetr}
\newpage
\onecolumn
\appendix
\section{Inverse Fourier Transform of $E(\omega)/i\omega$}
\label{ap:fourier}

First, we must develop some Fourier transforms of useful functions. Starting with the constant function, $x(t)=a$. Instead of directly evaluating the transform, we instead look at the inverse transform of $X(\omega)$,
\begin{equation*}
F^{-1}(X(\omega))=x(t)=a=\frac{1}{2\pi}\int_{-\infty}^{\infty}X(\omega)e^{i\omega t}d\omega.
\end{equation*}
Clearly, then $X(\omega)$ must be $2\pi a\delta(\omega)$. So,
\begin{equation}
a\rightleftharpoons 2\pi a\delta(\omega).
\end{equation}
Now, we examine the signum function,
\begin{equation}
\mathit{sgn}(t) =
    \begin{cases}
    +1~&\mathrm{if}~t>0 \\
    0~&\mathrm{if}~t=0  \\
    -1~&\mathrm{if}~t<0 \\
    \end{cases}.
\end{equation}
Which has the property
\begin{equation}
\frac{d}{dt}\mathit{sgn}(t) = 2\delta(t).
\end{equation}
Taking the Fourier transform of this function follows, using integration by parts
\begin{equation*}
\begin{aligned}
F(\mathit{sgn}(t))&= \int_{-\infty}^{\infty}\mathit{sgn}(t)e^{-i\omega t}dt \\
&= \left[\frac{-1}{i\omega}\mathit{sgn}(t)e^{-i\omega t}\right]_{-\infty}^{\infty} - \int_{-\infty}^{\infty}\frac{-2}{i\omega}\delta(t)e^{-i\omega t}dt \\
&= \frac{2}{i\omega}.
\end{aligned}
\end{equation*}
So,
\begin{equation}
\mathit{sgn}(t)\rightleftharpoons\frac{2}{i\omega}.
\end{equation}
The last function we need is the Heaviside step function,
\begin{equation}
H(t) =
    \begin{cases}
    1~&\mathrm{if}~t\geq0 \\
    0~&\mathrm{if}~t<0
    \end{cases}.
\end{equation}
Noticing that $H(t)=\frac{\mathit{sgn}(t)+1}{2}$ we take the Fourier transform of $H$,
\begin{equation*}
\begin{aligned}
F(H(t))&=F\left(\frac{\mathit{sgn}(t)+1}{2}\right) \\
&=F\left(\frac{\mathit{sgn}(t)}{2}\right)+F\left(\frac{1}{2}\right) \\
&= \frac{1}{i\omega}+\pi\delta(\omega).
\end{aligned}
\end{equation*}
So,
\begin{equation}
H(t)\rightleftharpoons\frac{1}{i\omega}+\pi\delta(\omega).
\end{equation}

Now, consider a general function, $f(t)$ which can be written as the integral from $-\infty$ to $t$ of some other function, $g(t)$,
\begin{equation}
f(t)=\int_{-\infty}^{t}g(\tau)d\tau.
\end{equation}
This can be expressed as a convolution integral with the Heaviside function,
\begin{equation}
f(t)=\int_{-\infty}^{t}g(\tau)d\tau = \int_{-\infty}^{\infty}g(\tau)H(t-\tau)d\tau=(g*H)(t).
\end{equation}
Fourier's convolution theorem states that for any two functions, $\theta(t)$ and $\phi(t)$, whose Fourier transforms exist,
\begin{equation}
F((\theta*\phi)(t)) = \Theta(\omega)\Phi(\omega).
\end{equation}
Applying this to our function, $f(t)$, yields,
\begin{equation}
F(f(t)) = G(\omega)\left[\frac{1}{i\omega}+\pi\delta(\omega)\right].
\end{equation}
So,
\begin{equation}
\int_{-\infty}^{t}g(\tau)d\tau\rightleftharpoons G(\omega)\left[\frac{1}{i\omega}+\pi\delta(\omega)\right].
\end{equation}
Finally, we can use this identity to find $\mathbf{E}(t)$, given $\mathbf{E}(\omega)$. We apply the above identity to each of the components of $\mathbf{E(\omega)}\left[\frac{1}{i\omega}+\pi\delta(\omega)\right]$. Then,
\begin{equation}
\int_{-\infty}^{t}\mathbf{E}(\tau)d\tau\rightleftharpoons \mathbf{E}(\omega)\left[\frac{1}{i\omega}+\pi\delta(\omega)\right].
\end{equation}
We may assume that $\mathbf{E}(t)$ is zero for $t<0$ and $\mathbf{E}(\omega=0)=0$, in which case the above equation reduces to the desired identity,
\begin{equation}
\int_0^t\mathbf{E}(\tau)d\tau\rightleftharpoons \frac{\mathbf{E}(\omega)}{i\omega}.
\end{equation}
\newpage
\section{The Lagrange Remainder Theorem}
\label{ap:lagrange}
\begin{theorem}
\label{thm:lagrange}
Let $f(x)$ be a function which is continuous on $[a,b]$ and $(n+1)$ differientiable on $(a,b)$. Then, for each $x\in[a,b]$ there exists $\eta\in[a,b]$ such that\footnote{The final term in this equation is called the Lagrange remainder of the $n^\mathrm{th}$ Taylor polynomial of $f$. This is because it is the error between $f$ and the $n^\mathrm{th}$ Taylor polynomial of $f$, $T_n$. That is to say, \\$f(x)-T_n(x) = \frac{f^{(n+1)}(\eta)}{(n+1)!}~\mathrm{for~some}~\eta\in[a,b]$}
\begin{equation}
f(x) = f(a) + f'(a)(x-a) + \frac{f''(a)}{2}(x-a)^2 + \cdots + \frac{f^{(n)}(a)}{n!}(x-a)^n + \frac{f^{(n+1)}(\eta)}{(n+1)!}(x-a)^{n+1}.
\label{eq:lagrange}
\end{equation}
\end{theorem}
\begin{proof}
Let $T_n$ be the $n^\mathrm{th}$ Taylor polynomial of $f$, centered at $a$. That is,
\begin{equation}
T_n(x) = f(a) + f'(a)(x-a) + \frac{f''(a)}{2}(x-a)^2 + \cdots + \frac{f^{(n)}(a)}{n!}(x-a)^n.
\end{equation}
Now choose $K\in\mathbb{R}$ such that
\begin{equation}
\label{eq:K}
f(b) = T_n(b) + K(b-a)^{n+1}.
\end{equation}
We will show that this $K$ is coefficient of the final term in Eq.~\ref{eq:lagrange}. First we consider the function $F$, defined as
\begin{equation}
F(x) = f(x) - T_n(x) - K(x-a)^{n+1}.
\end{equation}
Notice that $F^{(k)}(a)=0~\mathrm{for}~k\leq n$ since
\begin{equation}
T_n^{(k)}(x) = f^{(k)}(a) + f^{(k+1)}(a)(x-a) + \cdots + \frac{f^{(n)}(a)}{(n-k)!}(x-a)^{n-k}~\mathrm{for}~k\leq n.
\end{equation}
So $T_n^{(k)}(a)=f^{(k)}(a)$ for $k\leq n$. And,
\begin{equation}
\frac{d^k}{dx^k}K(x-a)^{n+1} = \frac{K(n+1)!}{(n+1-k)!}(x-a)^{n+1-k}.
\end{equation}
This demonstrates that $F^{(k)}(a)=0~\mathrm{for}~k\leq n$. Our choice of $K$ guarantees $F(b)=0$, so we see that $F(a)=F(b)=0$. This allows us to apply Rolle's theorem to $F$. The theorem is a special case of the mean value theorem, and it states that for any function, $g$, that is continuous on $[a,b]$ and differientiable on $(a,b)$, if $g(a)=g(b)$ then there exists at least one $c\in[a,b]$ such that $g'(c)=0$. Applying this theorem to $F$ tells us that there exists a $c_1\in[a,b]$ such that $F'(c_1)=0$. Now we examine $F'$. We have shown above that $F'(a)=0$, and we have just demonstrated that $F'(c_1)=0$, thus $F'(a)=F'(c_1)=0$. Applying Rolle's theorem now to $F'$ tells us that there exists $c_2\in[a,c_1]$ such that $F''(c_2)=0$. We continue this process on the higher derivatives of $F$ up to $F^{(n+1)}$. Here, Rolle's theorem guarantees the existence of a $c_{n+1}\in[a,c_n]$ such that $F^{(n+1)}(c_{n+1})=0$. However, since $T_n^{(n+1)}\equiv 0$, we know that $F^{(n+1)}$ is
\begin{equation}
F^{(n+1)}(x) = f^{(n+1)}(x) - K(n+1)!.
\end{equation}
So, using the fact that $F^{(n+1)}(c_{n+1})=0$,
\begin{equation}
K = \frac{f^{(n+1)}(c_{n+1})}{(n+1)!}.
\end{equation}
Let $\eta=c_{n+1}$. Since $[a,c_{n+1}]\subseteq[a,c_n]\subseteq[a,c_{n-1}]\subseteq\cdots\subseteq[a,b]$, we know $\eta\in[a,b]$. Moreover, since our choice of $K$ in Eq.~\ref{eq:K} could be repeated with $b$ replaced by any $x\in[a,b]$, the entire process could be done with any point in $[a,b]$, yielding a not-necessarily different $\eta$ for each point. However, each $\eta$ would still be in $[a,b]$ since $[a,x]\subseteq[a,b]~\forall~x\in[a,b]$. Plugging this expression for $K$ back into Eq.~\ref{eq:K} shows that for each $x\in[a,b]$, there exists $\eta\in[a,b]$ such that
\begin{equation}
f(x) = f(a) + f'(a)(x-a) + \frac{f''(a)}{2}(x-a)^2 + \cdots + \frac{f^{(n)}(a)}{n!}(x-a)^n + \frac{f^{(n+1)}(\eta)}{(n+1)!}(x-a)^{n+1}.
\end{equation}
\end{proof}
\end{document}
